\documentclass[10pt,a4paper]{article}
\usepackage[utf8]{inputenc}
\usepackage[OT4]{fontenc}
\usepackage{amsmath}
\usepackage{amsfonts}
\usepackage{amssymb}
\usepackage{titling}
\usepackage{anysize}
\usepackage{parskip}
\usepackage{graphicx}
\usepackage[polish]{babel}
\usepackage{float}
\usepackage[figurename=Wykres]{caption}
% some settings
\setlength{\droptitle}{-2cm}
\preauthor{}
\DeclareRobustCommand{\authorthing}{
\begin{center}
\begin{tabular}{cc}
\large{Jakub Woźniak} & \large{Marcin Zaremba} \\
109686 & 109702\\
jakub.l.wozniak@student.put.edu.pl & marcin.m.zaremba@student.put.edu.pl % takiego masz maila?
\end{tabular}\\
\vspace{8pt}
\large{Prowadzący: prof.\ dr hab.\ inż.\ Maciej Drozdowski}\\
Wydział Informatyki Politechniki Poznańskiej
\end{center}}
\author{\authorthing}
\date{8 listopada 2013}
\postauthor{}
\title{Job-Shop Scheduling\\Laboratorium Optymalizacji Kombinatorycznej}
\marginsize{2cm}{2cm}{1cm}{1cm}

\setlength{\parindent}{0pt}
% document
\begin{document}
\maketitle
\section{Wprowadzenie}
\subsection{Opis problemu}
Problem Job-Shop (JSP) jest problemem optymalizacji kobminatorycznej, który polega na szeregowaniu zadań (task) dla zleceń (job) przy wykorzystaniu ustalonej liczby maszyn. Każde zlecenie składa się z listy zadań, które muszą zostać wykonane w określonej kolejności, a każde zadanie składa się z numeru maszyny, na której ma zostać wykonane oraz czasu przetwarzania.\\
JSP należy do klasy problemów NP-trudnych, gdyż jest uogólnieniem problemu komiwojażera (który jest specjalnym przypadkiem JSP, gdzie liczba maszyn $m=1$, komiwojażer jest maszyną, a miasta, które odwiedza, to zlecenia).
\subsection{Program testujący}
\subsubsection{Opis}
Aplikacja została napisana w języku C i wykorzystuje biblioteki dostępne w systemach z rodziny GNU/Linux. Skrypt automatyzujący testowanie i generowanie wykresów został napisany w języku BASH.
\subsubsection{Kompilacja}
Kompilacja została zautomatyzowana przy pomocy programu GNU Make. Aby skompilować program należy z poziomu konsoli wydać polecenie \textit{make}. Prawidłowe zakończenie procesu kompilacji zostanie zasygnalizowane komunikatem \textit{Linking complete!}.
\subsubsection{Uruchomienie}
Automatyczny system testowania należy uruchomić przez wydanie polecenia \textit{make tests}.\\
Praca aplikacji jest sterowana przez przełączniki powszechnie stosowane w oprogramowaniu linuksowym. Przykładowe wywołanie \textit{bin/jobshop -t taillard -a greedy -f instances/tai10.txt -m} uruchomi program jobshop dla instancji Taillarda tai10.txt i wykona algorytm zachłanny (greedy). Wynik szeregowania zostanie zwrócony na standardowe wyjście, a jego ostatnią linią będzie czas działania algorytmu (przełącznik \textit{-m}). Pełna lista przełączników zostanie wyświetlona po wywołaniu programu bez żadnych parametrów.\\
Zapisanie wyniku działania programu do pliku odbywa się przy pomocy operatora przekierowania strumienia standardowego wyjścia dostępnego w powłoce bash: \textit{bin/jobshop -a greedy (\ldots) \textgreater results.txt}.
\section{Algorytm zachłanny}
Na potrzeby ćwiczenia opracowano i zaimplementowano heurystykę zachłanną generującą poprawne rozwiązania problemu szeregowania.

\subsection{Opis algorytmu}
Algorytm przechowuje w pamięci tablicę iteratorów reprezentujących aktualnie przetwarzane lub czekające na zasoby zadanie dla każdego ze zleceń oraz tablicę przechowującą aktualnie przetwarzane zadanie lub stan bezczynności dla każdej z maszyn. Dodatkowa zmienna $T=0$ określa chwilę czasu, w której analizujemy sytuację.


Poszukiwanie rozwiązania opiera się na pętli z warunkiem wykonania wszystkich zleceń, która wykonuje następujące kroki:
\begin{enumerate}
    \item dla każdej z maszyn: jeżeli maszyna jest bezczynna to na podstawie tablicy iteratorów znajdź zadanie przeznaczone dla tej maszyny (i zapisz czas jego rozpoczęcia) lub pozostań w stanie bezczynności,
    \item znajdź czas zadania, które kończy się najwcześniej i zapisz w zmiennej $t$,
    \item wykonaj $T += t$ i dla każdego aktualnie przetwarzanego zadania zmniejsz pozostały czas wykonania o $t$,
    \item jeżeli istnieje zadanie o pozostałym czasie wykonania równym 0, to usuń je z maszyny i przesuń iterator zlecenia o 1 w prawo.
\end{enumerate}

Po zakończeniu powyższej pętli długość uszeregowania znajduje się w zmiennej $T$. 

\subsection{Instancja testowa}
Algorytm został przetestowany przy pomocy dostarczonego oprogramowania \textit{chk\_jsorl.exe} dla wszystkich instancji w formacie orlib i Taillarda (po przekonwertowaniu do formatu orlib). Dla wszystkich rozwiązań oprogramowanie sprawdzające zwróciło wynik pozytywny.

\begin{table}[H]
    \begin{center}
        \begin{tabular}{ l l l l l l }
            3 & 3 &&&& \\
            0 & 2 & 1 & 2 & 2 & 2\\
            1 & 2 & 0 & 4 & 2 & 1\\
            2 & 4 & 1 & 2 & 0 & 1
        \end{tabular}
        \caption*{Tabela 1: przykładowa instancja testowa w formacie orlib}
        \label{test1}
    \end{center}
\end{table}

\begin{table}[H]
    \begin{center}
        \begin{tabular}{ l l l }
            7 & &\\
            0 & 2 & 4\\
            0 & 2 & 6\\
            0 & 4 & 6
        \end{tabular}
        \caption*{Tabela 2: wynik uszeregowania algorytmem zachłannym dla instancji zaprezentowanej w tabeli 1}
    \end{center}
\end{table}

\begin{figure}[H]
    \centering
    \includegraphics[scale=0.8]{gantt.pdf}
    \caption*{Rysunek 1: wykres Gantta dla uszeregowania podanego w tabeli 2}
\end{figure}

\section{Ocena efektywności}
Dla instancji tai20 --- tai25 dokonano pomiaru czasu wykonywania i jakości uszeregowania, co przedstawiono na poniższych wykresach.

\begin{figure}[H]
\centering
\includegraphics{eff1-1.pdf} 
\caption{długość uszeregowania w porównaniu do wartości najlepszych granic dla instancji Taillarda}
\label{eff11}
\end{figure}

\begin{figure}[H]
    \centering
    \includegraphics{eff1-2.pdf}
    \caption{czas wykonywania algorytmu dla zadanych instancji}
    \label{eff12}
\end{figure}

 
Dla instancji tai25 przeprowadzono pomiar czasu wykonywania algorytmów dla $n=1\ldots20$. Wyniki zaprezentowano na poniższym wykresie.

\begin{figure}[H]
    \centering
    \includegraphics{eff2.pdf}
    \caption{czas wykonywania algorytmu dla instancji tai25 przy zmiennej liczbie wczytanych zleceń}
\end{figure}

Przypadek pesymistyczny (kiedy w każdej sprawdzanej chwili czasu kończy się tylko jedno zadanie) ma złożoność obliczeniową rzędu $O(mn^2)$, gdzie: $m$ --- liczba maszyn, $n$ --- liczba zleceń. W przypadku optymistycznym (wszystkie aktualnie przetwarzane zadania kończą się w tej samej chwili) uzyskujemy złożoność $O(mn)$. 
    
\section{Wnioski}
Zaproponowana heurystyka zachłanna generuje rozwiązania dopuszczalne, ale dalekie od optimum. Metoda ta nie korzysta z pełnej informacji odczytanej z pliku i nie przewiduje przyszłego wykorzystania maszyn celem zoptymalizowania obciążenia. Zaimplementowany algorytm jest symulacją pracy maszyn, gdzie kolejne zadania zgłaszają żądanie zasobów po zakończeniu przetwarzania ich poprzednika.\\
Otrzymane w ten sposób rozwiązanie może posłużyć jako baza dla optymalizacji przy pomocy bardziej zaawansowanych algorytmów heurystycznych.
\end{document}
